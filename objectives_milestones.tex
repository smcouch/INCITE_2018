\section{Research Objectives and Milestones}
\label{sec:objectives}

Spark

M1

Pencil approach

RK3 approach

WENO5

1D CCSN comparison of central density

Convergence rate plots 

\vspace{0.1in} \noindent {\bf Year 1 --} Total Request: 100M {\it Mira} Service Units \vspace{-0.1in}


\subsection{Full 3D Multi-Dimensional Neutrino Transport Simulations}
\label{sec:3Dm1}

Additionally in Y1, we will carry out a small number of 3D CCSN simulations employing full multi-dimensional, energy-dependent neutrino transport.
These simulations will use the ``M1'' transport scheme of \citet{OConnor:2013ja} (see also \citep{Kuroda:2012fv}).
M1 computes the first two moments (the zeroth and first, hence ``M1'') of the Boltzmann equation for neutrinos with an analytic closure for the higher-order moments.
The neutrino fluxes are computed explicitly as a hyperbolic system resulting in favorable performance and scaling properties (at the cost of time step sizes limited by the speed of light) while the matter-radiation source terms are computed implicitly.
This implicit solve is completely local and requires solving only a 4x4 matrix.
Our initial implementation of M1 neglects velocity-dependent terms and inelastic scattering of neutrinos.
This can result in quantitative differences as compared to neutrino transport calculations that include such effects \citep[see][]{Lentz:2012fy} but for Y1 this is a necessary approximation that reduces the expense of the calculations.
In Y2 and Y3, we will include velocity-dependence and/or inelastic scattering in FLASH-M1.
These enhancements to our M1 scheme have already been incorporated into our 1D general relativistic code GR1D and our comparisons in 1D between the more advance transport and the simplified transport we propose for Y1 are qualitatively similar, if quantitatively different.
%We will include our subgrid model of the MRI in at least one M1 simulation.
%These M1 simulations will be truly state-of-the-art 3D CCSN calculations.

The M1 approach, thanks to solving for an additional, higher-order moment, is inherently more accurate than zeroth-moment only approaches such as flux-limited diffusion \citep[e.g.,][]{Bruenn:2013es, {Dolence:2014wp}}.
M1 does not require a flux-limiter-based closure for the radiation fluxes as they are solved for directly.
Furthermore, the analytic closure we currently use for the moments beyond the first is simple and straightforward yet shows encouraging agreement with 1D Monte Carlo neutrino transport calculations \citep{OConnor:2012phd}.
As compared with flux-limited diffusion, M1 does not suffer from the inability to capture ``shadows'' inherent to FLD schemes.
A known limitation of M1 is cases in which distinct beams of radiation intersect.
The M1 solution in such cases becomes highly diffuse at the intersection.
This is a problem in, e.g., radiation hydrodynamic calculations of accretion disks.
For CCSNe, however, the radiation field is highly forward peaked and cases in which distinct beams of radiation might cross are essentially non-existent.
Hence, M1 is {\it ideally} suited for the CCSN problem due to its accuracy (for the specific problem) and efficiency.
In addition, the severe limitation of time steps determined by the speed of light is not so drastic in CCSNe since the explicit time step is already just a factor of a few larger than this thanks to the enormous sound speeds in the PNS.
Another significant advantage of M1 is that it is a fully multidimensional transport scheme, i.e., the solution at a given grid point is dependent on the fluxes from every direction around that point.
This is distinct from the often-adopted ``ray-by-ray'' approximation \citep[e.g.,][]{Bruenn:2013es, {Muller:2012gd}, {Hanke:2013kf}} in which the transport problem is solved only along discrete radial rays.

We will use three neutrino species and 12 energy groups per species.
We have already conducted 1D and 2D comparisons between 12 energy groups and 18 energy groups and have found good agreement for our M1 transport scheme.
We have already fully implemented the 3D M1 method in FLASH and are running test simulations now.
Our performance studies of FLASH-M1 show that it is remarkably efficient compared to other neutrino transport schemes, owing largely to its explicit nature.
We find a usage rate for 3D FLASH-M1 simulations with 12 energy groups and three neutrino species of $5.5\times10^{-7}$ MSU per zone per step, only roughly 5$\times$ the usage rate of the MHD leakage simulations.
When factoring in the increase in the number of time steps due to the Courant condition now being set by the speed of light, an additional factor of about three must be included in the expense of the M1 simulations as compared to the leakage simulations (the speed of sound in the PNS is roughly $1/3$ the speed of light).

For Y1, we plan two 3D M1 CCSN simulations, one simulation each for 11.2-$M_\odot$ and 27-$M_\odot$ progenitors.
These two progenitors are selected to give points of comparison to 3D simulations of the Garching group using ray-by-ray neutrino transport but with inelastic scattering and velocity-dependence \citep{Hanke:2013kf, {Tamborra:2014wq}}.
Utilizing a finest grid spacing of 0.49 km and an effective angular resolution of $0.54^\circ$, each simulation will consist of a time-averaged zone count of $\sim$57.3 million zones and require over 600,000 time steps to cover 500 ms of post-bounce evolution.
At the measured usage rate for our 3D M1 simulations, the total cost estimate for each simulation is then 19.3M MSU.
With an additional request of 1.2M MSU for development, the total requested allocation for the Y1 M1 simulations is 40M MSU.




\vspace{0.1in} \noindent {\bf Year 2 --} Total Request: 100M {\it Mira} Service Units \vspace{-0.1in}


\subsection{3D MHD Parameter Study}
\label{sec:ParamStudy}

The MRI simulation of Y1 will be used to calibrate a sub-grid model for use in MHD CCSN simulations lacking sufficient resolution to capture the MRI directly.
This sub-grid model will include the amplification of large-scale magnetic fields in the linear regime and the dissipation of heat and transport of angular momentum due to unresolved turbulence once the MRI has reached saturation (an $\alpha$-viscosity prescription similar to \citet{Thompson:2005iw}).
Using this sub-grid model we will conduct a large parameter study of 3D magnetorotational CCSNe in which we will vary the progenitor model and the initial spin rate of the progenitor core.
Using this set of simulations, we will determine the degree to which progenitor rotation and magnetic fields aid neutrinos in driving supernova explosions for a broad range of stars.
We will include in these simulations realistic non-spherical velocity fluctuations, via the approach of \citet{Chatzopoulos:2014uj}.
Such perturbations have been shown to have an important, {\it qualitative} impact on the CCSN mechanism \citep{Couch:2013bl}.
We will study the resulting properties of newly-formed neutron stars such as spin, magnetic field strength and geometry, and kick velocity.
We will use these simulations to study the interplay between the SASI and neutrino-driven convection in aiding supernova shock expansion in the magnetorotational context.
We will also compute the gravitational wave signal from the supernovae using {\it in-situ} calculation of the quadrupole moments of the PNS.
The parameter study will consist of 7 simulations and for these simulations we will use both the 15 $M_\odot$ and 25 $M_\odot$ rotating, magnetic progenitors of \citet{Heger:2005bi}, or our own, if available.

\begin{comment}
Previous studies of MRCCSNe in 2D \citep{Burrows:2007gu} and 3D \citep{Winteler:2012fv} have amplified the initial magnetic fields so that the post-bounce fields are of order the estimated MRI-saturation field strengths.
Despite what may be reasonable final field strengths, by neglecting the exponential growth of the MRI, these previous studies get the wrong field growth history and geometry.
The proposed work will, for the first time, eliminate this arbitrary history of field growth associated with unphysical initial magnetic fields and subsequent unphysically slow growth rates.
We will employ a sub-grid model that approximates the amplification due to the MRI using the linear theory of the MRI fastest-growing mode.
Semi-global simulations in the context of CCSNe show that the overall growth of the MRI is dominated by the fastest-growing mode \citep{Obergaulinger:2009fv}.
Our sub-grid model will allow us to capture the time- and space-dependent amplification of the magnetic field due to the MRI in a realistic way without needing to resolve the length scale of the fastest-growing mode.
This model will be calibrated and tuned to match the results of the MRI-resolved simulation of Y1 (\S\ref{sec:mriSim}).
\end{comment}

We will use the MRCCSNe parameter study to explore the dependence of newly-formed neutron star parameters, such as spin, kick, and magnetic field strength and geometry, on progenitor characteristics.
We will calculate the gravitational wave emission and spherical harmonics of the shock.
These results will be compared with the non-rotating, non-magnetic models.
We will use a finest resolution of 0.5 km and an effective angular resolution of $0.45^\circ$.
Each simulation will cost 3.8M MSU.  For 7 simulations, we request 27M MSU plus an additional 1M MSU for testing.

\subsection{3D MHD Multi-Dimensional Neutrino Transport Sims}
\label{sec:enhancedM1}

In Y2, we will carry out three 3D MHD CCSN with M1 neutrino transport, as in Y1.
As of now, no 3D full transport CCSN simulation in the literature has included magnetic fields.
Additionally, we will include physically-motivated velocity fluctuations in the progenitor star \citep{Couch:2013bl, Chatzopoulos:2014uj}.
We have already shown that such fluctuations are important to enhancing the efficiency of the neutrino heating in CCSN, but they could also dramatically enhance the magnetic field amplification behind the stalled shock by enhancing the turbulent dynamo \citep[e.g.][]{Endeve:2012ht}.
For these simulations, we will take the 15-$M_\odot$, 20-$M_\odot$, and 25-$M_\odot$ progenitors of \citet{Heger:2005bi}.
Using the same resolution as in the Y1 M1 simulations, each simulation will cost 19M MSU bring the total request for these simulation to 57M MSU.

\subsection{Realistic CCSN Nucleosynthesis}
\label{sec:nucleo}

CCSNe are principally responsible for the production of elements heavier than Lithium throughout the universe.
The nucleosynthetic yields from CCSN simulations is a key quantity that can be directly compared to observations and laboratory measurements of cosmic abundances.
As such, we propose to compute detailed nucleosynthesis from our CCSN simulation results.
This will be accomplished as a post-processing step using large ($\sim$1000 isotopes) nuclear reaction networks furnished by Co-I's Arcones and Fr\"ohlich.
The input for the nuclear reaction networks will be passive tracer particle data that records thermodynamic trajectory information from our FLASH CCSN simulations.
FLASH already includes a well-developed, efficient passive particles framework that has been used extensively in the calculation of nucleosynthesis in Type Ia supernova simulations \citep[e.g.,][]{Long:2014dv}.

The nucleosynthesis during the first second post-bounce is interesting and scientifically valuable by itself, but we will also go beyond this short time.
In collaboration with Co-I Arcones and her graduate student, we have adapted our FLASH CCSN application to smoothly transition from the high-density NSE EOS to a reduced nuclear reaction network and appropriate EOS at low densities.
%FLASH includes several nuclear reaction networks to choose from, but in order to make the transition from the four-species NSE at high densities, we have added the ability to incorporate an additional neutron-rich tracer nucleus to the networks that allows us to match the $\bar A$ and $\bar Z$ of the NSE network making for a smooth EOS transition.
%At high densities, the multispecies approximate network is maintained in NSE through use of an appropriate NSE solver.
The ability to accurately simulate the nucleosynthesis and thermodynamics of low-density regions allows us to extend our CCSN simulations to late times and large radii, even to follow the development of explosions through the entire progenitor \citep[e.g.,][]{{Kifonidis:2003hs}, Couch:2009bu, {Couch:2011cf}}.
With these multidimensional simulations we will be able to take a remarkably detailed look at nucleosynthesis and mixing in the earliest stages of the formation of a young CCSN remnant \citep{Hammer:2010di}.
Comparison to observations of galactic CCSN remnants, such as Cas A \citep{Grefenstette:2014ds}, will be made as well as comparison to cosmic chemical abundances.

The post-processing nuclear reaction networks are embarrassingly parallel, requiring no interprocess communication.
The large number of particles that will be required for each 3D simulation ($\sim$millions) will easily make these Capability class simulations, though only short runtimes will be needed.
For this we request 5M MSU.
The 3D simulations of long timescale CCSN mixing, using the methods developed in \citet{Couch:2011cf}, will require a roughly equal amount of computing resources for each decade in radius simulated.
These simulations will not require neutrino transport and so are relatively inexpensive.
Restricting ourselves to compact progenitor stars ($R_* \sim 10^{6}$ km), we plan two 3D whole-star simulations each, requiring about 5M MSU.  The total request for the nucleosynthesis and 3D whole-star simulations is then 15M MSU.

\vspace{0.1in} \noindent {\bf Year 3 --} Total Request: 100M {\it Mira} Service Units\vspace{-0.1in}

\subsection{3D Multi-Dimensional Neutrino Transport Sims with Enhanced Physics}
\label{sec:enhancedM1}

In the third year of the project we will extend our M1 calculations to include velocity terms and inelastic scattering in the transport equations.
This will dramatically increase the expense of the simulations, but as \citet{Lentz:2012fy} point out, these terms can have an important impact on the results of CCSN simulations.
These calculations will push forward the leading edge of sophistication in CCSN simulation.
The realistic neutrino transport, coupled with our handling of the MRI, will make these the most physically-complete and accurate CCSN simulations yet attempted.
The coupled-energy group version of the M1 code is already developed in GR1D.
Early tests indicate that the expense is approximately four times the current M1 scheme.
%Given the great cost, we will conduct this simulation in reduced resolution.
We will simulate a rotating, magnetic progenitor and include the MRI sub-grid model.
For $dx_{\rm min} = 0.5$ km and effective angular resolution of $0.55^\circ$ the simulation will consist of 53 million zones and 600,000 time steps to simulation 500 ms of post-bounce evolution, costing 70M MSU.
We request a further 5M MSU for testing and development of the energy-coupled M1 code.

\subsection{Monte Carlo Radiative Transfer of 3D CCSN Simulations}
\label{sec:radtrans}

\todo{Talk about Sedona and Kasen?}

In Y3 of this project, we will make direct comparison to EM observations of CCSNe through the calculation of light curves and spectra from our simulations.
We will utilize the newly-developed Implicit Monte Carlo/Discrete Diffusion Monte Carlo radiative transfer code, SuperNu \citep{Wollaeger:2013ix}.
SuperNu is presently being extended to 3D by utilizing the very same AMR grid package as FLASH, making compatibility between the two codes straight-forward.
We will use as input for these calculations the hydrodynamic and nucleosynthetic results of the previous two years.
Being a Monte Carlo method, SuperNu is inherently embarrassingly parallel, though testing in 1D indicates a very large number of Monte Carlo particle packets are necessary for good signal-to-noise.
Thus we anticipate these calculations to be expensive for 3D data and request an allocation of 25M MSU for our Y3 3D radiative transfer calculations.


% \begin{wrapfigure}[17]{r}{3.5in}
% \includegraphics[width=3.7in, trim= 0in 0in 0in .27in, clip]{data/m25}
% \caption{MRI characteristic parameters from a 1.5D collapse simulation of a 25 $M_{\odot}$ star at 70 ms post-bounce.
% Within the PNS ($\lesssim 40$ km) entropy and composition gradients stabilize the MRI.}
% \label{fig:mri}
% \end{wrapfigure}
