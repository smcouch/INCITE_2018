\documentclass[12pt,letterpaper]{article}
\usepackage[T1]{fontenc}
\usepackage[pagestyles,raggedright]{titlesec}
\usepackage{txfonts}
\usepackage{blindtext}
\usepackage{xcolor}
\usepackage{fancyhdr}
\usepackage[top=1in, bottom=1in, left=1in, right=1in] {geometry}
\usepackage{graphicx}

\input macros.tex

% Set one inch margins:
\setlength{\textwidth}{6.5in}
\setlength{\textheight}{9.0in}
\setlength{\oddsidemargin}{0.0cm}
\setlength{\evensidemargin}{0.0cm}
\setlength{\topmargin}{-0.5in}
\setlength{\headheight}{0.25in}
\setlength{\headsep}{0.25in}

\pagestyle{fancy}


\begin{document}

\setlength{\parindent}{0in}

\pagestyle{fancy}
%\lhead{\doctitle{}}
%\rhead{S.M. Couch}
\renewcommand{\headrulewidth}{0.0pt}

\begin{center} \textbf{\doctitle{}} \\
%  A Proposal to the DOE ASCR Leadership Computing Challenge
\end{center}

\begin{flushleft}
\textbf{PI}:
Sean M. Couch, University of Chicago \\
\medskip

\textbf{Executive Summary:}

{\parindent 16pt

  Core-collapse supernovae (CCSNe) are the luminous explosions that
  herald the death of massive stars.  Neutron stars, pulsars,
  magnetars, and stellar-mass black holes are all born out of these
  explosions.  Some Gamma-Ray Bursts (GRBs) have been associated with
  CCSNe, raising the possibility of a common progenitor for both.
  CCSNe are responsible for the production of many elements throughout
  the universe, especially those heavier than iron; their importance
  in galactic chemical evolution cannot be underestimated.  Many of
  the first stars, expected to be relatively massive, likely ended as
  CCSNe as well.  These bright events, occurring just a couple of
  million years after the Big Bang, may be some of the most distant,
  observable objects in the universe with the upcoming James Webb
  Space Telescope.  Despite the importance of CCSNe to our
  understanding of many aspects of the universe the mechanism that
  reverses stellar core collapse and drives supernova explosions is
  not fully understood leaving our understanding of many important
  astrophysical phenomena incomplete.  The CCSN mechanism is one of
  the most important challenges for modern computational astrophysics.
  Leadership-class computing is the only path forward toward solving
  this long-standing problem.

  We propose to conduct a comprehensive study of the impact of
  rotation and magnetic fields on core-collapse supernovae over a
  three-year period on the BG/Q {\it Mira} at the Argonne Leadership
  Computing Facility (ALCF). We will carry out a series of 3D
  magnetohydrodynamics (MHD) simulations of the collapse of rotating,
  magnetic stellar cores using the FLASH multi-physics, adaptive mesh
  refinement (AMR) simulation framework.  We will include in our
  simulations realistic treatments for neutrino physics, realistic
  progenitor rotation and magnetic fields, and, for the first time
  ever, sufficient resolution to capture the growth of the
  magnetorotational instability (MRI).  We will use these simulations
  to quantify how much rotational energy of the progenitor cores can
  be tapped to aid neutrinos in driving successful and robust
  explosions.  Our simulations will allow us to predict the spins,
  kicks, magnetic field strengths and alignments of newly-formed
  neutron stars (NSs), pulsars, and magnetars, as well as the
  dependence of these parameters on progenitor conditions.  Using {\it
    in-situ} analysis, we will compute the gravitational wave signal
  from our simulations, providing valuable theoretical data for
  experiments such as Advanced LIGO.  Our simulations will be the most
  physically-detailed and accurate CCSN simulations to include
  magnetorotational effects ever accomplished, with the potential for
  uncovering a robust and realistic CCSN explosion mechanism.

}
\end{flushleft}
\end{document}
