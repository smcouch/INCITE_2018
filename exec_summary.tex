\documentclass[12pt,letterpaper]{article}
\usepackage[T1]{fontenc}
\usepackage[pagestyles,raggedright]{titlesec}
\usepackage{txfonts}
\usepackage{blindtext}
\usepackage{xcolor}
\usepackage{fancyhdr}
\usepackage[top=1in, bottom=1in, left=1in, right=1in] {geometry}
\usepackage{graphicx}

\input macros.tex

% Set one inch margins:
\setlength{\textwidth}{6.5in}
\setlength{\textheight}{9.0in}
\setlength{\oddsidemargin}{0.0cm}
\setlength{\evensidemargin}{0.0cm}
\setlength{\topmargin}{-0.5in}
\setlength{\headheight}{0.25in}
\setlength{\headsep}{0.25in}

\pagestyle{fancy}


\begin{document}

\setlength{\parindent}{0in}

\pagestyle{fancy}
%\lhead{\doctitle{}}
%\rhead{S.M. Couch}
\renewcommand{\headrulewidth}{0.0pt}

\begin{center} \textbf{\doctitle{}} \\
%  A Proposal to the DOE ASCR Leadership Computing Challenge
\end{center}

\begin{flushleft}
\textbf{PI}:
Sean M. Couch, Michigan State University \\
\medskip

\textbf{Executive Summary:}

{\parindent 16pt

Core-collapse supernovae (CCSNe) are the most extreme laboratories for nuclear physics in the universe.
Stellar core collapse and the violent explosions that follow give birth to neutron stars and black holes, and in the process synthesize most of the elements heavier than helium throughout the universe.
The behavior of matter at supranuclear densities is crucial to the CCSN mechanism, as are strong and weak interactions.
Beyond Standard Model behavior of neutrinos may also impact the CCSN mechanism.
Despite the key role CCSNe play in many aspects of astrophysics, and decades of research effort, {\it we still do not fully understand the details of the physical mechanism that causes these explosions.}
This leaves frustratingly large error bars on many key aspects of our theoretical understanding of the universe, and also makes it difficult to constrain uncertain nuclear physics with data from CCSNe.

We propose an end-to-end, multi-year investigation of CCSNe that includes the effects of rotation, magnetic fields, and progenitor asphericity.
Our comprehensive research program will consist of 3D MHD CCSN simulations with sophisticated multi-dimensional neutrino transport, the most realistic initial conditions ever adopted for the study of CCSNe, and an intensive comparison to observations through the calculation of gravitational wave emission, detailed nucleosynthesis, and electromagnetic radiative transfer.
The ambitious objectives of this project will be achievable by leveraging the unique combination of skills in the proposal team, cutting-edge open-source software, and the Leadership-class resources available through the INCITE program.

We propose a multi-year progressive investigation of the CCSN mechanism using realistic initial conditions.
This project will develop and employ 3D massive stellar progenitor models at the point of core-collapse, including rotation and magnetic fields.
We will address the critically important questions of whether rotation and magnetic fields aid successful explosions for ``normal'' CCSNe and how rotation and magnetic fields effect the nucleosynthesis in CCSNe.
Our results will directly inform our understanding of the characteristics of newborn pulsars and magnetars, information that can be directly compared to observational data.

Our project will address two critical questions: How do plausible rotation rates and magnetic field strengths influence the CCSN mechanism? and What is the impact of realistic 3D progenitor structure including rotation and magnetic fields on the CCSN mechanism and observables?

}
\end{flushleft}
\end{document}
